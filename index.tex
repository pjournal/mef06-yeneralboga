% Options for packages loaded elsewhere
\PassOptionsToPackage{unicode}{hyperref}
\PassOptionsToPackage{hyphens}{url}
\PassOptionsToPackage{dvipsnames,svgnames,x11names}{xcolor}
%
\documentclass[
  letterpaper,
  DIV=11,
  numbers=noendperiod]{scrreprt}

\usepackage{amsmath,amssymb}
\usepackage{lmodern}
\usepackage{iftex}
\ifPDFTeX
  \usepackage[T1]{fontenc}
  \usepackage[utf8]{inputenc}
  \usepackage{textcomp} % provide euro and other symbols
\else % if luatex or xetex
  \usepackage{unicode-math}
  \defaultfontfeatures{Scale=MatchLowercase}
  \defaultfontfeatures[\rmfamily]{Ligatures=TeX,Scale=1}
\fi
% Use upquote if available, for straight quotes in verbatim environments
\IfFileExists{upquote.sty}{\usepackage{upquote}}{}
\IfFileExists{microtype.sty}{% use microtype if available
  \usepackage[]{microtype}
  \UseMicrotypeSet[protrusion]{basicmath} % disable protrusion for tt fonts
}{}
\makeatletter
\@ifundefined{KOMAClassName}{% if non-KOMA class
  \IfFileExists{parskip.sty}{%
    \usepackage{parskip}
  }{% else
    \setlength{\parindent}{0pt}
    \setlength{\parskip}{6pt plus 2pt minus 1pt}}
}{% if KOMA class
  \KOMAoptions{parskip=half}}
\makeatother
\usepackage{xcolor}
\setlength{\emergencystretch}{3em} % prevent overfull lines
\setcounter{secnumdepth}{5}
% Make \paragraph and \subparagraph free-standing
\ifx\paragraph\undefined\else
  \let\oldparagraph\paragraph
  \renewcommand{\paragraph}[1]{\oldparagraph{#1}\mbox{}}
\fi
\ifx\subparagraph\undefined\else
  \let\oldsubparagraph\subparagraph
  \renewcommand{\subparagraph}[1]{\oldsubparagraph{#1}\mbox{}}
\fi


\providecommand{\tightlist}{%
  \setlength{\itemsep}{0pt}\setlength{\parskip}{0pt}}\usepackage{longtable,booktabs,array}
\usepackage{calc} % for calculating minipage widths
% Correct order of tables after \paragraph or \subparagraph
\usepackage{etoolbox}
\makeatletter
\patchcmd\longtable{\par}{\if@noskipsec\mbox{}\fi\par}{}{}
\makeatother
% Allow footnotes in longtable head/foot
\IfFileExists{footnotehyper.sty}{\usepackage{footnotehyper}}{\usepackage{footnote}}
\makesavenoteenv{longtable}
\usepackage{graphicx}
\makeatletter
\def\maxwidth{\ifdim\Gin@nat@width>\linewidth\linewidth\else\Gin@nat@width\fi}
\def\maxheight{\ifdim\Gin@nat@height>\textheight\textheight\else\Gin@nat@height\fi}
\makeatother
% Scale images if necessary, so that they will not overflow the page
% margins by default, and it is still possible to overwrite the defaults
% using explicit options in \includegraphics[width, height, ...]{}
\setkeys{Gin}{width=\maxwidth,height=\maxheight,keepaspectratio}
% Set default figure placement to htbp
\makeatletter
\def\fps@figure{htbp}
\makeatother

\KOMAoption{captions}{tableheading}
\makeatletter
\makeatother
\makeatletter
\@ifpackageloaded{bookmark}{}{\usepackage{bookmark}}
\makeatother
\makeatletter
\@ifpackageloaded{caption}{}{\usepackage{caption}}
\AtBeginDocument{%
\ifdefined\contentsname
  \renewcommand*\contentsname{Table of contents}
\else
  \newcommand\contentsname{Table of contents}
\fi
\ifdefined\listfigurename
  \renewcommand*\listfigurename{List of Figures}
\else
  \newcommand\listfigurename{List of Figures}
\fi
\ifdefined\listtablename
  \renewcommand*\listtablename{List of Tables}
\else
  \newcommand\listtablename{List of Tables}
\fi
\ifdefined\figurename
  \renewcommand*\figurename{Figure}
\else
  \newcommand\figurename{Figure}
\fi
\ifdefined\tablename
  \renewcommand*\tablename{Table}
\else
  \newcommand\tablename{Table}
\fi
}
\@ifpackageloaded{float}{}{\usepackage{float}}
\floatstyle{ruled}
\@ifundefined{c@chapter}{\newfloat{codelisting}{h}{lop}}{\newfloat{codelisting}{h}{lop}[chapter]}
\floatname{codelisting}{Listing}
\newcommand*\listoflistings{\listof{codelisting}{List of Listings}}
\makeatother
\makeatletter
\@ifpackageloaded{caption}{}{\usepackage{caption}}
\@ifpackageloaded{subcaption}{}{\usepackage{subcaption}}
\makeatother
\makeatletter
\@ifpackageloaded{tcolorbox}{}{\usepackage[many]{tcolorbox}}
\makeatother
\makeatletter
\@ifundefined{shadecolor}{\definecolor{shadecolor}{rgb}{.97, .97, .97}}
\makeatother
\makeatletter
\makeatother
\ifLuaTeX
  \usepackage{selnolig}  % disable illegal ligatures
\fi
\IfFileExists{bookmark.sty}{\usepackage{bookmark}}{\usepackage{hyperref}}
\IfFileExists{xurl.sty}{\usepackage{xurl}}{} % add URL line breaks if available
\urlstyle{same} % disable monospaced font for URLs
\hypersetup{
  pdftitle={Yener Alboga Progress Journal},
  colorlinks=true,
  linkcolor={blue},
  filecolor={Maroon},
  citecolor={Blue},
  urlcolor={Blue},
  pdfcreator={LaTeX via pandoc}}

\title{Yener Alboga Progress Journal}
\author{}
\date{}

\begin{document}
\maketitle
\ifdefined\Shaded\renewenvironment{Shaded}{\begin{tcolorbox}[interior hidden, boxrule=0pt, enhanced, borderline west={3pt}{0pt}{shadecolor}, breakable, sharp corners, frame hidden]}{\end{tcolorbox}}\fi

\renewcommand*\contentsname{Table of contents}
{
\hypersetup{linkcolor=}
\setcounter{tocdepth}{2}
\tableofcontents
}
\bookmarksetup{startatroot}

\hypertarget{introduction}{%
\chapter*{Introduction}\label{introduction}}
\addcontentsline{toc}{chapter}{Introduction}

This progress journal covers Yener Alboğa's work during their term at
\href{https://mef-bda503.github.io/fall22/}{BDA 503 Fall 2022}.

Each section is an assignment or an individual work.

\bookmarksetup{startatroot}

\hypertarget{assignment-1}{%
\chapter{Assignment 1}\label{assignment-1}}

Hello all ! I am Yener Alboğa . Although I entered 2009, I graduated
from Boğaziçi University Industrial Engineering in 2020(No typo just 5
years deviation from expected graduation year:) Besides, I have about 9
years of professional work experience. The first 3-4 years of this are
in the supply chain. After that, I worked in Quick commerce and
e-commerce companies for catalogue\& category management analysis.
Currently, I am working as a consultant on catalogue management to a
company called Zoodmall, which is trying to sell products from Turkey
and China as B2C to MENA and CIS countries. At the same time, I am a
seller who tries to sell products with fulfilled by merchant and Retail
arbitrage model on eBay, AliExpress and Amazon. With the development of
my Data Science skills, I aim to carry out productive and creative work
in companies that I will work in the field of e-commerce in the medium
term. In the long term, I aim to have a business that I can manage
independently of location from anywhere in the world by establishing an
e-commerce model with a high level of efficiency.

\href{https://www.linkedin.com/in/yener-alboga/}{My LinkedIn Profile}

\hypertarget{user-2022-video}{%
\section{UseR-2022 Video}\label{user-2022-video}}

An Introduction to the Apple Health Exports speech by
\href{https://www.johngoldin.com/}{John Goldin} is interesting. Although
I currently believe that smartwatches are the devices that leave us
doomed to more notifications and chargers, I believe that they will have
great support in health in the near future.

\href{https://www.rstudio.com/conference/2022/talks/introduction-to-apple-health-export/}{UseR-2022
John Goldin Talk Details}

\hypertarget{some-interesting-r-posts}{%
\section{Some Interesting R Posts}\label{some-interesting-r-posts}}

I think it is very fun and valuable to be able to analyse with R and
twitter. It's important to be able to do this as an influencer or as a
policy follower etc. We can see why which tweet gets the most engagement
and how it reaches larger audiences or in which time frame it will get
more engagement etc. (I don't know fancy wording for peak time of social
media but for radio channels it's called ``drive time''.)

\href{https://www.toptal.com/r/social-network-analysis-in-r-gephi-tutorial}{Social
Network Analysis}

There is a recommendation library to analyse data in R which can help us
to upsell or recommend new song that we think the user might like by
looking at their historical data. \#recommenderlab Netflix has
established a recommendation system using more than 20000 variables.
These variables consider the viewing habits of users with similar
characteristics, as well as variables such as user data, region and the
year of production of the film's actors.

\href{https://www.data-mania.com/blog/how-to-build-a-recommendation-engine-in-r/}{Recommendation
Engine}

We understood the importance of logistics even better during the
pandemic and the global crisis that followed. Logistics systems have
very heavy costs. Therefore, finding the optimum is very valuable. If we
explain this with current examples, the decision of Getir, Yemeksepeti
and Bisu dark store locations or Trendyol Express distribution points is
of great importance in increasing the efficiency of the warehouse.
(Example: Rental expenses, ease of access to the supplier or distance to
the customer) Below link has very basic calculation for finding/deciding
warehouse locations but some with maps libraries and variables we can
develop a great GIS tool.

\href{https://www.supplychaindataanalytics.com/monte-carlo-simulation-in-r-for-warehouse-location-risk-assessment/}{Monte
Carlo Simulation for Warehouse Allocation Using R}

Thank You !

\hypertarget{my-progress-journal-home-page}{%
\section{My Progress Journal Home
Page}\label{my-progress-journal-home-page}}

\href{https://pjournal.github.io/mef06-yeneralboga/}{Home}

\bookmarksetup{startatroot}

\hypertarget{assignment-2format-and-output-to-be-updated}{%
\chapter{Assignment 2(Format and output to be
updated}\label{assignment-2format-and-output-to-be-updated}}

install.packages(``nycflights13'') install.packages(``tidyverse'')
install.packages(``dplyr'') nycflights13::planes

planes \%\textgreater\% group\_by(manufacturer) \%\textgreater\%
summarise(count=n()) \%\textgreater\% arrange(desc(count))
\%\textgreater\% print(n=Inf)

\bookmarksetup{startatroot}

\hypertarget{if-we-want-to-see-how-many-flights-held-by-which-manufacturers-planes-before-year-2000-with-more-than-2-engines}{%
\chapter{If we want to see how many flights held by which manufacturers'
planes before year 2000 with more than 2
engines}\label{if-we-want-to-see-how-many-flights-held-by-which-manufacturers-planes-before-year-2000-with-more-than-2-engines}}

planes\%\textgreater\% filter(year\textless2000 \& engines
\textgreater2)\%\textgreater\% group\_by(manufacturer)\%\textgreater\%
summarize(count=n())\%\textgreater\%
arrange(desc(count))\%\textgreater\% print(n=Inf)

\bookmarksetup{startatroot}

\hypertarget{or-case}{%
\chapter{OR Case}\label{or-case}}

Yener Alboga\\
2023-01-03

\hfill\break

\textbf{German First Division Basketball League Scheduling Case Study}

We have seen in real time during the Covid period how unexciting sports
events are when played without fans. In addition, a large part of the
sports economy consists of support from fans and TV broadcasting
revenues. In addition, creating a schedule where athletes can rest as
much as possible and as evenly as possible is very important for both
athlete health and fair competition.

\hypertarget{constraints}{%
\section{Constraints}\label{constraints}}

The fact that each team should not play consecutive matches in their own
field or away, that the match times of the big teams should not overlap
with each other, that the teams that have matches outside the local
league on weekdays can rest as equally as possible between the 3 matches
they will play in a week are among the important constraints of this
case. And there are other constraints like matching the most interesting
games to available TV slots, minimizing total driving distances.

\hypertarget{result}{%
\section{Result}\label{result}}

As an example, let's take the team of Bayern Munich, which also plays in
the Euroleague. Let's assume that the Beko Basketball League matches are
played on Friday, Saturday, and Sunday. Likewise, Euroleague matches are
played on Thursday - Friday. In this case, it would not be healthy for
Bayern Münich to play the Euroleauge game on Thursday and then play the
domestic league game on Friday. At the same time, if he played the
Euroleague game away, he must play the domestic league game at home so
that he has the opportunity to rest at the maximum level. This was just
a regular 1 week example for 1 team.

Now lets consider that Basketball Bundesliga is the premier basketball
league in Germany with 18 teams and over 300 games each season.

And with correct objective function \& constraints The German First
Division Basketball League finds scheduling solutions in less than 20
minutes for previously unsolvable models.

\hypertarget{case-source}{%
\section{Case Source}\label{case-source}}

You can reach the original case source below

\href{https://www.gurobi.com/case_studies/german-first-division-basketball-league-scheduling/}{Case
Source}



\end{document}
